\title{Taxi Trajectory Prediction}
\author{
        Karl Krauth (z3416790)\\
        David McKinnon (z3421068)
}
\date{\today}

\documentclass[12pt]{article}

\begin{document}
\maketitle

\section{Introduction}
Taxis nowadays use electronic dispatch systems for scheduling pick-ups, but they do not usually enter their drop-off locations. Therefore, when a call comes for a taxi, it is difficult for dispatchers to know which taxi to contact. 

To improve the efficiency of electronic taxi dispatching systems it is important to be able to predict the final destination of a taxi while it is in service. Since there is often a taxi whose current ride will end near a requested pick up location from a new passenger, it would be useful to know approximately where each taxi is likely to end so that the system can identify the best taxi to assign to each new pickup request. This lowers the waiting time for new passengers and allows the taxi system to operate more efficiently.

This project occurs in the context of a Kaggle competition hosted by ECML/PKDD. The goal of this competition is to predict the destination of taxis travelling in Porto, Portugal given specific data about the current trip. To aid with this a dataset of around 1.7 million complete trips is provided.

\end{document}
